\section{Discussions and concluding remarks} \label{sec:discussion}

Our model demonstrates that BERT with a layer of logistical regression could successfully extract the supporting phrase of a tweet with \score\ loss. We also tested adding a Question and Answering layer in between BERT and logistical regression, where the context is the tweet, question is one of positive, neutral or negative, and the answer is  the supporting phrase. However due to time limitations, we were not able to train this new model and evaluate its performance. However, we do learn that some Kaggle contestants achieved a lower loss using this method. We have not tested the model with other variants of BERT. 

It is also interesting to explore whether there is a method that can
extract supporting phrases that are not continuous.

\subsection{The Data}
For the most part, the supporting phrases provided in the training set are reasonable.  

However, on closer inspection, a portion of the supporting phrases given as ground
truth in the training set are problematic.  Some of these problems are
given in the following table.

\begin{table}[!h]
	\centering
	\begin{tabular} {c | p{5cm} l c}
	Tweet ID & Tweet & Supporting phrase & Sentiment \\
	\hline
          799f &  Yes it does. Please dont` go. If you die I will
                 cry.. which normally leads to small animals getting
                 harmed & g harmed & negative \\
          \hline
          a54d & I know   It was worth a shot, though! & as wort &
                                                                   positive \\
          \hline
	5bd5 & You make me happy, whether you know it or not  <3 & You make me happy, wh & positive
	\end{tabular}
	\caption{Sample of problematic training set}
\end{table} 

The organizers of the competition did not offer an explanation as to
why this happens.  It could be the result of human error, since the
error usually involves including a few letters from the previous or
next word.  Sometimes the selection itself is questionable, too.  For the
first tweet in the table, the supporting phrase would arguably be ``I
will cry'' rather than ``harmed''.

Also, when the sentiment is ``neutral'', in 90\% of the cases the
supporting phrases is just the whole sentence.  For the rest 10\%, it
is often the sentence but with the URL inside them removed.
Predicting according to this rule improved our score somewhat.

%%% Local Variables:
%%% mode: latex
%%% TeX-master: t
%%% End:
