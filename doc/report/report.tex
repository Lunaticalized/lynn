\documentclass{article}

% if you need to pass options to natbib, use, e.g.:
%     \PassOptionsToPackage{numbers, compress}{natbib}
% before loading neurips_2018

% ready for submission
% \usepackage{neurips_2018}

% to compile a preprint version, e.g., for submission to arXiv, add add the
% [preprint] option:
%     \usepackage[preprint]{neurips_2018}

% to compile a camera-ready version, add the [final] option, e.g.:
\usepackage[final]{nips_2018}

% to avoid loading the natbib package, add option nonatbib:
%     \usepackage[nonatbib]{neurips_2018}

\usepackage[utf8]{inputenc} % allow utf-8 input
\usepackage[T1]{fontenc}    % use 8-bit T1 fonts
\usepackage{hyperref}       % hyperlinks
\usepackage{url}            % simple URL typesetting
\usepackage{booktabs}       % professional-quality tables
\usepackage{amsfonts}       % blackboard math symbols
\usepackage{nicefrac}       % compact symbols for 1/2, etc.
\usepackage{microtype}      % microtypography
\usepackage{amsmath}
\usepackage{graphicx}
\newcommand{\score}{100}
\title{Tweet Sentiment Extraction}

% The \author macro works with any number of authors. There are two commands
% used to separate the names and addresses of multiple authors: \And and \AND.
%
% Using \And between authors leaves it to LaTeX to determine where to break the
% lines. Using \AND forces a line break at that point. So, if LaTeX puts 3 of 4
% authors names on the first line, and the last on the second line, try using
% \AND instead of \And before the third author name.

\author{%
  Jingcheng Xu \\
  % examples of more authors
  \And
  Xiating Ouyang \\
  % Affiliation \\
  % Address \\
  % \texttt{email} \\
  % \AND
  % Coauthor \\
  % Affiliation \\
  % Address \\
  % \texttt{email} \\
  % \And
  % Coauthor \\
  % Affiliation \\
  % Address \\
  % \texttt{email} \\
  % \And
  % Coauthor \\
  % Affiliation \\
  % Address \\
  % \texttt{email} \\
}

\begin{document}
% \nipsfinalcopy is no longer used

\maketitle

\begin{abstract}
	Understanding users' sentiments is a crucial task for certain industries whose business model heavily relies on the feedback from users. Sentiment analysis classifies the sentiment expressed in a given text into sentiment labels such as positive, negative and neutral, and such classification can often be directly supported by a phrase within the text, called the supporting phrase. In this report, we present a supporting phrase extraction model using BERT that finds the supporting phrase in a given a text that supports the sentiment label \cite{devlin2018bert}. 
\end{abstract}

\section{Introduction} \label{sec:introduction}

Sentiment analysis is a critical task for industries involving product recommendation or social behavior prediction. In an Internet era, the volume of information explodes exponentially and it becoming increasingly inpractical for the industries to study each feedback or opinion from its customer manually. Twitter has become a popular social media with roughly 6,000 tweets sent per second, corresponding to 500M tweets sent per day. This huge volume of tweets allows the use of machine learning for industry owners to automatically classify the sentiments of the tweet senders into positive, neutral or negative, so that the industry owners could understand the users' sentiments for their business planning. 

Support phrase extraction extends beyond the classification task---given the sentiment of the tweet, it also asks for the specific phrase within the tweet that supports the classified sentiment. This allows a more fine-grained representation of the tweet senders' sentiments. Currently there is an undergoing \href{https://www.kaggle.com/c/tweet-sentiment-extraction}{Kaggle competition} for support phrase extraction: Given a set tweets, each with its true emotional label, find the supporting phrase within each tweet supporting its sentiment. 


Our solution uses distill BERT (Bidirectional Encoder Representations from Transformers), a state-of-the-art tool for sentiment analysis \cite{devlin2018bert} \cite{sanh2019distilbert}. 


\paragraph{Organization} In Section~\ref{sec:model} we describe the input dataset and our model; In Section~\ref{sec:experiment} we present the experimental results of our model; and finally in Section\ref{sec:discussion} we discuss some limitations of our model.


\section{Model} \label{sec:model}

\section{Experiments} \label{sec:experiment}

The performance is evaluated by the averaged Jaccard score. The Jaccard score between two sentences $s_1$ and $s_2$, both represented by an array of words, is defined as $$score(s_1, s_2) = \frac{|s_1 \cap s_2|}{|s_1 \cup s_2|}.$$ The test set contains 3,534 tweets samples.

Our experiment on the Kaggle dataset initially achieved an averaged Jaccard score of 0.491, using our model presented in Section~\ref{sec:model}. We later improved the model with the following preprocessing and tweeks:

\begin{itemize}
	\item Preprocess all tweets such that all urls, links and @ references are removed from the training set; and
	\item For tweets with neutral sentiments, we simply return the entire string.  
\end{itemize}

The averaged Jaccard score is improved to 0.564.
 
We are also constructing a model in which we train two separate models on the positive sentiments and the negative sentiments respectively, and then apply the corresponding trained model based on the test sample sentiment. 

\section{Discussions and concluding remarks} \label{sec:discussion}

Our model demonstrates that BERT with a layer of logistical regression could successfully extract the supporting phrase of a tweet with \score\ loss. We also tested adding a Question and Answering layer in between BERT and logistical regression, where the context is the tweet, question is one of positive, neutral or negative, and the answer is  the supporting phrase. However due to time limitations, we were not able to train this new model and evaluate its performance. However, we do learn that some Kaggle contestants achieved a lower loss using this method. We have not tested the model with other variants of BERT. 

It is also interesting to explore whether there is a method that can
extract supporting phrases that are not continuous.

\subsection{The Data}
For the most part, the supporting phrases provided in the training set are reasonable.  

However, on closer inspection, a portion of the supporting phrases given as ground
truth in the training set are problematic.  Some of these problems are
given in the following table.

\begin{table}[!h]
	\centering
	\begin{tabular} {c | p{5cm} l c}
	Tweet ID & Tweet & Supporting phrase & Sentiment \\
	\hline
          799f &  Yes it does. Please dont` go. If you die I will
                 cry.. which normally leads to small animals getting
                 harmed & g harmed & negative \\
          \hline
          a54d & I know   It was worth a shot, though! & as wort &
                                                                   positive \\
          \hline
	5bd5 & You make me happy, whether you know it or not  <3 & You make me happy, wh & positive
	\end{tabular}
	\caption{Sample of problematic training set}
\end{table} 

The organizers of the competition did not offer an explanation as to
why this happens.  It could be the result of human error, since the
error usually involves including a few letters from the previous or
next word.  Sometimes the selection itself is questionable, too.  For the
first tweet in the table, the supporting phrase would arguably be ``I
will cry'' rather than ``harmed''.

Also, when the sentiment is ``neutral'', in 90\% of the cases the
supporting phrases is just the whole sentence.  For the rest 10\%, it
is often the sentence but with the URL inside them removed.
Predicting according to this rule improved our score somewhat.

Our code is available at \url{https://github.com/Lunaticalized/lynn}.
%%% Local Variables:
%%% mode: latex
%%% TeX-master: t
%%% End:



\bibliography{reference}{}
\bibliographystyle{plain}

\end{document}